\documentclass[10pt,portrait]{article}

\usepackage[portrait]{geometry}
\usepackage[T1]{fontenc}
\usepackage[utf8x]{inputenc}
\usepackage[english]{babel}

\begin{document}
  \footnotesize
  \pagestyle{empty}
  
    \begin{center}
      \Huge{How to win friends and influence people - Dale Carnegie}
    \end{center}
    
    \section*{Fundamentals techniques in handling people}
    \begin{itemize}
      \item Don't criticise, condemn or complain.
      \item Give honest and sincere appreciation.
      \item Arouse in the other person an eager want.
    \end{itemize}
    
    \section*{Six ways to make people like you}
    \begin{itemize}
      \item Become genuinely interested in other people.
      \item Smile.
      \item Remember that a person's name is to that person the sweetest and most important sound in any language.
      \item Be a good listener. Encourage others to talk about themselves.
      \item Talk in terms of the other person's interests.
      \item Make the other person feel important - and do it sincerely.
    \end{itemize}
    
    \section*{Win people to your way of thinking}
    \begin{itemize}
      \item The only way to get the best of an argument is to avoid it.
      \item Show respect for the other person's opinions. Never say, 'You're wrong'. 
      \item If you are wrong, admit it quickly and emphatically.
      \item Begin in a friendly way.
      \item Get the other person saying 'yes, yes' immediately.
      \item Let the other person do a great deal of the talking.
      \item Let the other person feel that the idea is his or hers.
      \item Try honestly to see things from the other person's point of view.
      \item Be Sympathetic with the other person's ideas and desires.
      \item Appeal to the nobler motives.
      \item Dramatise your ideas.
      \item Throw down a challenge.
    \end{itemize}
    
    \section*{Be a leader}
    \begin{itemize}
      \item Begin with praise and hoset appreciation.
      \item Call attention to people mistakes indirectly.
      \item Talk about your own mistakes before criticising the other person.
      \item Ask questions instead of giving direct orders.
      \item Let the other person save face.
      \item Praise the slightest improvement and praise every improvement. Be 'hearty in your approbation and lavish in your praise'.
      \item Give the other person a fine reputation to live up to.
      \item Use encouragement. Make the fault seem easy to correct.
      \item Make the other person happy about doing the thing you suggest.
    \end{itemize}
\end{document}