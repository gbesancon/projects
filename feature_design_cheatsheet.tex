\documentclass[10pt,landscape]{article}

\usepackage{multicol}
\usepackage{calc}
\usepackage{url}
\usepackage{ifthen}
\usepackage[landscape]{geometry}
\usepackage[hidelinks]{hyperref}
\usepackage[T1]{fontenc}
\usepackage[utf8x]{inputenc}
\usepackage[english]{babel}

% To make this come out properly in landscape mode, do one of the following
% 1.
%  pdflatex latexsheet.tex
%
% 2.
%  latex latexsheet.tex
%  dvips -P pdf  -t landscape latexsheet.dvi
%  ps2pdf latexsheet.ps

% This sets page margins to .5 inch if using letter paper, and to 1cm
% if using A4 paper. (This probably isn't strictly necessary.)
% If using another size paper, use default 1cm margins.
\ifthenelse{\lengthtest { \paperwidth = 11in}}
	{ \geometry{top=.5in,left=.5in,right=.5in,bottom=.5in} }
	{\ifthenelse{ \lengthtest{ \paperwidth = 297mm}}
		{\geometry{top=1cm,left=1cm,right=1cm,bottom=1cm} }
		{\geometry{top=1cm,left=1cm,right=1cm,bottom=1cm} }
	}

% Turn off header and footer
\pagestyle{empty}

% Redefine section commands to use less space
\makeatletter
\renewcommand{\section}{\@startsection{section}{1}{0mm}%
                                {-1ex plus -.5ex minus -.2ex}%
                                {0.5ex plus .2ex}%x
                                {\normalfont\large\bfseries}}
\renewcommand{\subsection}{\@startsection{subsection}{2}{0mm}%
                                {-1explus -.5ex minus -.2ex}%
                                {0.5ex plus .2ex}%
                                {\normalfont\normalsize\bfseries}}
\renewcommand{\subsubsection}{\@startsection{subsubsection}{3}{0mm}%
                                {-1ex plus -.5ex minus -.2ex}%
                                {1ex plus .2ex}%
                                {\normalfont\small\bfseries}}
\makeatother

% Define BibTeX command
\def\BibTeX{{\rm B\kern-.05em{\sc i\kern-.025em b}\kern-.08em
    T\kern-.1667em\lower.7ex\hbox{E}\kern-.125emX}}

% Don't print section numbers
\setcounter{secnumdepth}{0}

\setlength{\parindent}{0pt}
\setlength{\parskip}{0pt plus 0.5ex}

% -----------------------------------------------------------------------

\begin{document}
  \raggedright
  \footnotesize
  \begin{multicols}{4}[
    \begin{center}
      \Huge{Feature name}
    \end{center}
  ]
  
    % multicol parameters
    % These lengths are set only within the two main columns
    %\setlength{\columnseprule}{0.25pt}
    \setlength{\premulticols}{1pt}
    \setlength{\postmulticols}{1pt}
    \setlength{\multicolsep}{1pt}
    \setlength{\columnsep}{2pt}

    \section*{Description}
    \begin{itemize}
      \item Objectives
    \end{itemize}
    
    \section*{Traceability}
    \begin{itemize}
      \item Requirements
      \item User stories (\href{http://www.agilemodeling.com/artifacts/userStory.htm}{link})
      \item Technical documentation
      \item Contact
    \end{itemize}
  
    \section*{User Interface}
    \begin{itemize}
      \item Command Line Interface
      \begin{itemize}
        \item Arguments
      \end{itemize}
      \item Graphical User Interface
      \begin{itemize}
        \item Prototype
        \item User interactions
      \end{itemize}
    \end{itemize}
  
    \section*{Architecture}
    \begin{itemize}
      \item Technologies
      \begin{itemize}
        \item Languages
        \item Third parties
      \end{itemize}
      \item Component diagram
      \begin{itemize}
        \item Dependencies
        \item Interface
        \item Data formats
        \item Interaction
      \end{itemize}
      \item Error/Exception management
    \end{itemize}

    \section*{Design}
    \begin{itemize}
      \item Evolutivity
      \item Genericity
      \item Limitations
      \item Class diagram
      \begin{itemize}
        \item Role of classes
        \item Design-patterns
      \end{itemize}
      \item Sequence diagram
      \item Configuration
      \begin{itemize}
        \item Settings
        \item Provider (file, database, ...)
      \end{itemize}
    \end{itemize}

    \section*{Implementation}
    \begin{itemize}
      \item Processing
      \begin{itemize}
        \item State diagram
        \item Algorithm
        \begin{itemize}
          \item Cases to implement
        \end{itemize}
      \end{itemize}
      \item Data
      \begin{itemize}   
        \item Size (min / average / max)
      \end{itemize}
      \item Performance
      \begin{itemize}
        \item Execution time
        \item Memory consumption
      \end{itemize}
    \end{itemize}
  
    \section*{Testing}
    \begin{itemize}
      \item{Use-cases}
      \begin{itemize}
        \item Given \dots When \dots Then \dots (\href{http://guide.agilealliance.org/guide/gwt.html}{link})
        \item Use-case diagram
      \end{itemize}    
      \item Strategy
      \begin{itemize}
        \item Mock-up
        \item Integrated
      \end{itemize}              
      \item Execution
      \begin{itemize}
        \item Manual tests
        \item Automatic tests
      \end{itemize}
      \item Test types
      \begin{itemize}
        \item Unit tests (Nominal/Off cases)
        \item Functional tests
        \item Integration tests
      \end{itemize}
      \item Coverage
      \begin{itemize}
        \item Percentage expected
      \end{itemize}
    \end{itemize}
    
    \section*{Deployment}
    \begin{itemize}
      \item 
    \end{itemize}

    \scriptsize
    Source: (\href{https://github.com/gbesancon/cheatsheet}{link})\\
    Generation date: \today\\
    Author: Gilles Besan\c{c}on\\
    \href{https://github.com/gbesancon/cheatsheet}{GitHub} \href{https://www.linkedin.com/in/gbesancon}{LinkedIn}\\
  \end{multicols}
\end{document}
